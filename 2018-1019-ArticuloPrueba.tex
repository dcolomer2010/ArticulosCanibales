\documentclass[12pt,a4paper,twoside,twocolumn,final]{article}
\usepackage[utf8]{inputenc}
\usepackage[spanish]{babel}
\usepackage{amsmath}
\usepackage{amsfonts}
\usepackage{amssymb}
\usepackage{makeidx}
\usepackage{graphicx}
\usepackage[left=2.00cm, right=1.00cm, top=3.00cm, bottom=2.00cm]{geometry}
%\usepackage{courier}
%\renewcommand*\familydefault{\ttdefault} %% Only if the base font of the document is to be typewriter style
%\usepackage[T1]{fontenc}
\usepackage{CormorantGaramond}
%\renewcommand*\familydefault{\ttdefault} %% Only if the base font of the document is to be typewriter style
%\usepackage[scaled]{beramono}
%\renewcommand*\familydefault{\ttdefault} %% Only if the base font of the document is to be typewriter style
\usepackage[T1]{fontenc}
\author{David Colomer Rosel}
\title{Hermanos de barro}
\begin{document}
	\maketitle
	
	El d\'ia comenz\'o como cualquier otro d\'ia anodino y repetitivo. Una taza de caf\'e recalentado del d\'ia anterior y un par de magdalenas y la misma m\'usica de siempre saliendo de la misma radio de siempre.
	
	La c\'amara lista junto con la libreta y el l\'apiz que me acompa\~{n}an siempre no desentonaban en esta rutina de dirigirme al destino de ese d\'ia. 
	
	Hoy hay una cosa distinta en el coche. Reluciente puedo ver mi casco con la palabra ``PRESS'' resaltando en blanco sobre el azul oc\'eano del casco. Como cualquier ni\~{n}o con zapatos nuevos no puedo aguantarme y me lo pongo tras depositar la mochila en el asiento trasero. 
	
	Un vecino entra en ese momento en el garaje y me saluda con una sonrisa c\'omplice. Otra excentricidad m\'as a incluir en mi curriculum vitae vecinal.
	
	La carretera est\'a despejada y el viaje es anodino. Me dirijo a las instalaciones de pruebas y entrenamiento de HCA\footnote{Holocausto Canibal Airsoft}. Hoy tienen preparada una sesi\'on para que aprenda a coordinarme mejor con sus movimientos.
	
	
	
	Puede que el avisado lector comprenda ya que este que les escribe es un periodista empotrado. Alguien que debe aprender a moverse con unidades en mitad ``del jaleo'' y no entorpecer ni incrementar el riesgo (propio y ajeno) cuando las cosas se ponen feas.
	
	
	
	Estos chicos con los que voy a compartir experiencias son como usted o como yo, pero con la particularidad de saber que hacer cuando los dem\'as perdemos la calma y el control. No son superhombres, pero sus capacidades y habilidades entrenadas hacen que sean capaces de responder en situaciones en las que los dem\'as nos vendr\'iamos abajo.
	
	La sesi\'on comienza con una charla introductoria sobre los objetivos que deben cubrirse en la sesi\'on. No hay margen para la improvisaci\'on. El responsable al mando, el TL, nos explica que es lo que estamos buscando: aprender a coordinarnos.
	
	Las caras de los m\'as veteranos muestran el gesto de qui\'en sabe que como dec\'ia el General Patton, ``no hay plan que aguante el primer disparo''. Los m\'as j\'ovenes est\'an tan atentos como yo para no perderse la m\'as m\'inima informaci\'on.
	
	He estado tomando notas, o al menos eso es lo que cre\'ia. Cuando releo lo apuntado aquello no tiene el m\'as m\'inimo sentido: flanco, retaguardia, voz de alerta, canal, radio, binomio.
	
	Algo me queda claro: soy en este momento una clase de apestado. Un elemento inc\'omodo que no saben como manejar. Un veterano me indica que ``tu conmigo'' con un tono de voz que no me da opci\'on a la discrepancia.
	
	El sol est\'a m\'as alto, pero no me calienta. Sin embargo estoy sudando, un efecto secundario de la gran cantidad de adrenalina que est\'a en este momento recorriendo mis venas.
	
	Mi c\'amara se contrapone a los fusiles de asalto. En una suerte de competencia f\'alica siento que la c\'amara no es lo suficientemente ``grande''. Me siento desnudo. Aprieto la rosca del objetivo, la montura, como buscado que crezca, entoces aprecio que ninguno est\'a tocando su arma. Estas cuelgan a plomo desde unas correas, no a la espalda como en aquellas pel\'iculas de la sobremesa del s\'abado. Me siento m\'as desnudo a\'un.
	
	Describir la cantidad de peso que estos seguidores de Marte portan es otra tarea dif\'icil. T\'erminos como ``molle'', ``pouch'', ``hop'', ``efepeeses'' y similares salen de su boca como entiendo que ``isos'', ``diafragmas'', ``exposici\'on'' deben de salir de la mia. Procuro no parecer muy perdido, si eso es posible.
	
	
	
\end{document}